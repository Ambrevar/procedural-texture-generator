%%%%%%%%%%%%%%%%%%%%%%%%%%%%%%%%%%%%%%%%%%%%%%%%%%%%%%%%%%%%%%%%%%%%%%%%%%%%%%%%
\documentclass[10pt,a4paper]{article}
\usepackage[utf8]{inputenc}
\usepackage[T1]{fontenc}
\usepackage{amsmath,amssymb,amsfonts}
\usepackage{geometry}
\usepackage{lmodern}
\usepackage{marvosym}
\usepackage{textcomp}
\DeclareUnicodeCharacter{20AC}{\EUR{}}
\DeclareUnicodeCharacter{2264}{\leqslant}
\DeclareUnicodeCharacter{2265}{\geqslant}
%%=============================================================================
%% Properties

\title{Random Number Generation}
\author{\textsc{P.~Neidhardt}}

%%=============================================================================
%% Aliases

\usepackage{xspace}

\let\latexbak\LaTeX
\renewcommand{\LaTeX}{\textrm{\latexbak}\xspace}

\let\texbak\TeX
\renewcommand{\TeX}{\textrm{\texbak}\xspace}

\def\unix{\textsc{Unix}\xspace}
\def\ie{\textsl{i.e.}\xspace}
\def\eg{\textsl{e.g.}\xspace}

%%=============================================================================
%% Formatting

% \usepackage{parskip}
% \setlength{\parindent}{15pt}

% \renewcommand{\thefigure}{\arabic{section}.\arabic{figure}}
\renewcommand{\arraystretch}{1.4}
% \renewcommand{\familydefault}{\sfdefault}

% \usepackage{needspace}

%%==============================================================================
%% Tables

% \usepackage{longtable}
% \usepackage{tabu}

%%==============================================================================
%% Graphics

%% Load TikZ after xcolor.
\usepackage[svgnames]{xcolor}
% \usepackage{graphicx}
% \usepackage{tikz}

% \newcommand{\fancybox}[1]{
%   \begin{tikzpicture}
%     \node[draw,rounded corners]{#1};
%   \end{tikzpicture}
% }

%%==============================================================================
%% Misc.

% \usepackage{calc}
% \usepackage{fp}
% \usepackage{lipsum}

%%=============================================================================
%% Listings

% \usepackage{listings}

%% Source code.
% \lstdefinestyle{customc}{
%   % numbers=left,
%   belowcaptionskip=1\baselineskip,
%   breaklines=true,
%   frame=L,
%   xleftmargin=\parindent,
%   % framexleftmargin=\parindent,
%   language=C,
%   showstringspaces=false,
%   basicstyle=\footnotesize\ttfamily,
%   keywordstyle=\bfseries\color{green!40!black},
%   commentstyle=\itshape\color{purple!40!black},
%   identifierstyle=\color{blue},
%   stringstyle=\color{orange},
%   numberstyle=\ttfamily
% }

% \lstset{escapechar=@,style=customc}

%%=============================================================================
%% Babel

%% Load last before 'hyperref'.
%% \usepackage[frenchb]{babel}

%%==============================================================================
%% Hyperref

%% Load last.
\usepackage[]{hyperref}

\hypersetup{
  colorlinks=true,
  linkcolor=DarkRed,
  linktoc=page,
  urlcolor=blue,
}

\def\N{\mathbb{N}}

%%%%%%%%%%%%%%%%%%%%%%%%%%%%%%%%%%%%%%%%%%%%%%%%%%%%%%%%%%%%%%%%%%%%%%%%%%%%%%%%
\begin{document}
%%%%%%%%%%%%%%%%%%%%%%%%%%%%%%%%%%%%%%%%%%%%%%%%%%%%%%%%%%%%%%%%%%%%%%%%%%%%%%%%
\maketitle
\vfill
\thispagestyle{empty}
%% \tableofcontents


\section{Random number generators}

\subsection{Maximal period}

Problem context: a random number generator (RNG) is an application from $\N$
to $\N^\N$, \ie{} given an integer it will return an integer sequence.

In computer science, numbers have an upper bound. RNG are often defined by a
recursive linear congruential sequence:
\[
u_{n+1} = (a \cdot u_n + b) \bmod m
\]
Here, $m$ is the upper bound. The sequence is periodic.
If the period is less than $m$, then the sequence values will not match all
values between $0$ and $m-1$.


\paragraph{Definition}
A RNG is said to be of \emph{maximal period} if all values of the codomain
are met by the sequence.

\paragraph{Property}

\[
\text{The period is maximal if and only if}
\left\{
  \begin{array}{l}
    \text{(1) } \gcd(b,m) = 1 \\
    \text{(2) } \text{For all primes $p$ dividing $m$, we have } a \bmod p = 1 \\
    \text{(3) } \text{If 4 divides $m$, then } a \bmod 4 = 1 \\
  \end{array}
\right.
\]



\paragraph{Proof}

I did not manage to prove this completely. Feel free to share your knowledge on
the problem.

\subparagraph{Relation (1)}

Let's assume that the period is maximal. The sequence can be solved:
\[
u_{n+1} = a \cdot u_n + b
\]
Fixed point:
\[
l = a \cdot l + b \Rightarrow l = \frac{b}{1-a}
\]
\[
u_{n+1} - l = a \cdot (u_n - l)
\]
\[
u_n = a^n \cdot (u_0 - l) + l
\]
The modulus does not change anything. Hence
\[
\boxed{ u_n = a^n \cdot (u_0 - l) + l \bmod m }
\]
By changing the indices, we can see this is equivalent to using a sequence where
the first term would be 0.
\[
u_n =  l \cdot (1 - a^n) \bmod m
\]
Since the period is maximal, we have $k \in [0;m-1]$ so that  $u_k = 1$, \ie
\[
u_k = 1 = l \cdot (1 - a^k) \bmod m
\]
\[
1 = \frac{b}{1-a} \cdot (1 - a^k) + A\cdot m
\]
with A an integer.
\[
\boxed{ 1 = b \cdot \sum_{i=0}^{k-1} a^i + A\cdot m }
\]
From Bézout's identity we can state that $b$ and $m$ are coprime.

\subparagraph{Relation (2)}
I got from an unknown source that first we have to prove if the period is
maximal for $m$, then it is maximal for $p$ dividing $m$. Hence
\[
u_{n+1} = (a \cdot u_n + b) \bmod p
\]
would be of maximal period too.

N.B.: if $a=1$, then
\[
u_{n+1} = (u_n + b) \bmod p
\]
And the period is effectively maximal if and only if $m$ and $b$ are coprime.
%% TODO: need to cite the theorem on finite groups that states that.

\subparagraph{Relation (3)}
According to relation (2) which is supposed proven,
\[
4 \text{ divise } m \Rightarrow a \bmod 4 = 1 \text{ ou } a \bmod 4 = -1
\]
If $a \bmod 4 = -1$, then
\[
u_{n+2} = -(-u_n + b) + b \bmod m = u_n
\]
The period is not maximal, so necessarily we have $a \bmod 4 = 1$.


%%%%%%%%%%%%%%%%%%%%%%%%%%%%%%%%%%%%%%%%%%%%%%%%%%%%%%%%%%%%%%%%%%%%%%%%%%%%%%%%
\end{document}
%%%%%%%%%%%%%%%%%%%%%%%%%%%%%%%%%%%%%%%%%%%%%%%%%%%%%%%%%%%%%%%%%%%%%%%%%%%%%%%%


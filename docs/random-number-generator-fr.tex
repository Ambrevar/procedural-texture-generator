%%%%%%%%%%%%%%%%%%%%%%%%%%%%%%%%%%%%%%%%%%%%%%%%%%%%%%%%%%%%%%%%%%%%%%%%%%%%%%%%
\documentclass[10pt,a4paper]{article}
\usepackage[utf8]{inputenc}
\usepackage[T1]{fontenc}
\usepackage{amsmath,amssymb,amsfonts}
\usepackage{geometry}
\usepackage{lmodern}
\usepackage{marvosym}
\usepackage{textcomp}
\DeclareUnicodeCharacter{20AC}{\EUR{}}
\DeclareUnicodeCharacter{2264}{\leqslant}
\DeclareUnicodeCharacter{2265}{\geqslant}
%%=============================================================================
%% Properties

\title{Random Number Generation}
\author{\textsc{P.~Neidhardt}}

%%=============================================================================
%% Aliases

\usepackage{xspace}

\let\latexbak\LaTeX
\renewcommand{\LaTeX}{\textrm{\latexbak}\xspace}

\let\texbak\TeX
\renewcommand{\TeX}{\textrm{\texbak}\xspace}

\def\unix{\textsc{Unix}\xspace}
\def\ie{\textsl{i.e.}\xspace}
\def\eg{\textsl{e.g.}\xspace}

%%=============================================================================
%% Formatting

% \usepackage{parskip}
% \setlength{\parindent}{15pt}

% \renewcommand{\thefigure}{\arabic{section}.\arabic{figure}}
\renewcommand{\arraystretch}{1.4}
% \renewcommand{\familydefault}{\sfdefault}

% \usepackage{needspace}

%%==============================================================================
%% Tables

% \usepackage{longtable}
% \usepackage{tabu}

%%==============================================================================
%% Graphics

%% Load TikZ after xcolor.
\usepackage[svgnames]{xcolor}
% \usepackage{graphicx}
% \usepackage{tikz}

% \newcommand{\fancybox}[1]{
%   \begin{tikzpicture}
%     \node[draw,rounded corners]{#1};
%   \end{tikzpicture}
% }

%%==============================================================================
%% Misc.

% \usepackage{calc}
% \usepackage{fp}
% \usepackage{lipsum}

%%=============================================================================
%% Listings

% \usepackage{listings}

%% Source code.
% \lstdefinestyle{customc}{
%   % numbers=left,
%   belowcaptionskip=1\baselineskip,
%   breaklines=true,
%   frame=L,
%   xleftmargin=\parindent,
%   % framexleftmargin=\parindent,
%   language=C,
%   showstringspaces=false,
%   basicstyle=\footnotesize\ttfamily,
%   keywordstyle=\bfseries\color{green!40!black},
%   commentstyle=\itshape\color{purple!40!black},
%   identifierstyle=\color{blue},
%   stringstyle=\color{orange},
%   numberstyle=\ttfamily
% }

% \lstset{escapechar=@,style=customc}

%%=============================================================================
%% Babel

%% Load last before 'hyperref'.
\usepackage[frenchb]{babel}

%%==============================================================================
%% Hyperref

%% Load last.
\usepackage[]{hyperref}

\hypersetup{
  colorlinks=true,
  linkcolor=DarkRed,
  linktoc=page,
  urlcolor=blue,
}

\DeclareMathOperator{\pgcd}{PGCD}

\def\N{\mathbb{N}}

%%%%%%%%%%%%%%%%%%%%%%%%%%%%%%%%%%%%%%%%%%%%%%%%%%%%%%%%%%%%%%%%%%%%%%%%%%%%%%%%
\begin{document}
%%%%%%%%%%%%%%%%%%%%%%%%%%%%%%%%%%%%%%%%%%%%%%%%%%%%%%%%%%%%%%%%%%%%%%%%%%%%%%%%
\maketitle
\vfill
\thispagestyle{empty}
%% \tableofcontents

%%%%%%%%%%%%%%%%%%%%%%%%%%%%%%%%%%%%%%%%%%%%%%%%%

\section{Générateur de nombres pseudo-aléatoires}

\subsection{Periode maximale}

Situation du problème: un générateur de nombre pseudo-aléatoire (GNPA) est une application de $\N$ dans $\N^\N$, i.e. qui à un entier renvoie une suite d'entiers.
En informatique, les nombres ont une borne supérieure. On définit souvent le GNPA par une suite récurrente congruentielle linéaire:
\[
u_{n+1} = (a \cdot u_n + b) \bmod m
\]
Ici, $m$ représente donc la borne supérieure. La suite définie précédemment est périodique.
Si la période est inférieure à $m$, alors la suite ne prendra pas toutes les valeurs comprises entre $0$ et $m-1$.

\paragraph{Définition}
Un GNPA est dit de \textbf{période maximale} si toutes les valeurs de son intervalle d'arrivée sont prises par la suite.

\paragraph{Propriété}

\[
\text{La période est maximale si et seulement si}
\left\{
  \begin{array}{l}
    \text{(1) } \pgcd(b,m) = 1 \\
    \text{(2) } \text{Pour tout $p$ premier divisant $m$, on a } a \bmod p = 1 \\
    \text{(3) } \text{Si 4 divise $m$, alors } a \bmod 4 = 1 \\
  \end{array}
\right.
\]



\paragraph{Démonstration: sens direct}
\subparagraph{Relation (1)}


Supposons la période max. Résolution de la suite arithmético-géométrique :
\[
u_{n+1} = a \cdot u_n + b
\]
Point fixe:
\[
l = a \cdot l + b \Rightarrow l = \frac{b}{1-a}
\]
\[
u_{n+1} - l = a \cdot (u_n - l)
\]
\[
u_n = a^n \cdot (u_0 - l) + l
\]
L'ajout du modulo ne change rien. On a donc
\[
\boxed{ u_n = a^n \cdot (u_0 - l) + l \bmod m }
\]
Par changement d'indice, on peut se ramener à une suite dont le premier terme est 0.
\[
u_n =  l \cdot (1 - a^n) \bmod m
\]
Comme la période est max, il existe $k \in [0;m-1]$ tel que $u_k = 1$, i.e.
\[
u_k = 1 = l \cdot (1 - a^k) \bmod m
\]
\[
1 = \frac{b}{1-a} \cdot (1 - a^k) + A\cdot m
\]
avec A entier relatif.
\[
\boxed{ 1 = b \cdot \sum_{i=0}^{k-1} a^i + A\cdot m }
\]
D'après le théorème de Bachet-Bézout, $b$ et $m$ sont premiers entre eux.


\subparagraph{Relation (2)}
D'après un étudiant en master, il faut d'abord montrer que si la période est maximale pour $m$, alors elle est maximale pour $p$ premier divisant $m$. Donc
\[
u_{n+1} = (a \cdot u_n + b) \bmod p
\]
serait de période max aussi.

N.B.: si $a=1$, alors
\[
u_{n+1} = (u_n + b) \bmod p
\]
Et la période est bien maximale si et seulement si $m$ et $b$ sont premiers entre eux d'après je ne sais plus quel théorème vu en spé sur les groupes finis.

\subparagraph{Relation (3)}
D'après la relation (2) (que l'on suppose démontrée),
\[
4 \text{ divise } m \Rightarrow a \bmod 4 = 1 \text{ ou } a \bmod 4 = -1
\]
Si $a \bmod 4 = -1$, alors
\[
u_{n+2} = -(-u_n + b) + b \bmod m = u_n
\]
La période n'est pas maximale, donc nécéssairement $a \bmod 4 = 1$.


%%%%%%%%%%%%%%%%%%%%%%%%%%%%%%%%%%%%%%%%%%%%%%%%%%%%%%%%%%%%%%%%%%%%%%%%%%%%%%%%
\end{document}
%%%%%%%%%%%%%%%%%%%%%%%%%%%%%%%%%%%%%%%%%%%%%%%%%%%%%%%%%%%%%%%%%%%%%%%%%%%%%%%%

